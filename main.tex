% This example is meant to be compiled with lualatex or xelatex
% The theme itself also supports pdflatex
\PassOptionsToPackage{unicode}{hyperref}
\documentclass[aspectratio=1610, 9pt]{beamer}

% Load packages you need here
\usepackage{polyglossia}
\setmainlanguage{german}

\usepackage{csquotes}
    

\usepackage{amsmath}
\usepackage{amssymb}
\usepackage{mathtools}

\usepackage{hyperref}
\usepackage{bookmark}

% load the theme after all packages

\usetheme[
  showtotalframes, % show total number of frames in the footline
]{tudo}

% Put settings here, like
\unimathsetup{
  math-style=ISO,
  bold-style=ISO,
  nabla=upright,
  partial=upright,
  mathrm=sym,
}

\title{Extending ctapipe image reconstruction using FACT methods}
\author[L Nickel/M.~Nöthe]{Lukas Nickel and Maximilian Nöthe}
%\institute[Kurzform Lehrstuhl]{Names des Lehrstuhls \\  Name der Fakultät}
%\titlegraphic{\includegraphics[width=0.7\textwidth]{images/tudo-title-2.jpg}}


\begin{document}

\maketitle

\begin{frame}{Overview}
 \begin{enumerate}
   \item What is CTA?
   \item How can FACT contribute?
   \item What changes/improves?
 \end{enumerate}
\end{frame}

\begin{frame}{Einführung}
  \tableofcontents
\end{frame}

\begin{frame}
    - was ist cta?
    - was ist geplant?
    - was ist fertig?
    - womit können wir arbeiten?
    - status MC
    - was ist ctapipe?
    - wie soll es funktionieren? low-level-kram -> ? daten level? dl0,dl1,...?
    - status ctapipe
\end{frame}

\begin{frame}{column test}
    \begin{columns}[T] % align columns
        \begin{column}{.48\textwidth}
            \color{red}\rule{\linewidth}{4pt}
            \begin{enumerate}
                \item "Cherenkov Telescope Array"
                \item Proposed in 2005, currently in pre-production
                \item Two arrays of multiple telescopes (>100) instead of single telescopes
                \item Goals: Extend observable energy range(20GeV-300TeV), huge field of view()
                \item Status: First light on LST and Schwarzschildt-Couder-Telescope
            \end{enumerate}
        \end{column}%
        \hfill%
        \begin{column}{.48\textwidth}
            \color{blue}\rule{\linewidth}{4pt}
            \begin{figure}
                \includegraphics[width=\linewidth]{images/cta_telescopes.jpg}
                \caption{Visualization of the different telescope types. Credit: CTA/M-A. Besel/IAC (G.P. Diaz)/ESO}
            \end{figure}
        \end{column}%
    \end{columns}
\end{frame}


\begin{frame}{Expected sensitivity}
    \begin{figure}
        \includegraphics[width= 0.8\linewidth]{images/cta_sensitivity.png}
        \caption{Credit: CTA/M-A. Besel/IAC (G.P. Diaz)/ESO}
    \end{figure}

\end{frame}

\begin{frame}{ctapipe}
    \begin{enumerate}
        \item low level pipeline for cta data
        \item calibration, cleaning, hillas, ...
        \item in development, prototype
        \item python based
        \item ??
    \end{enumerate}
\end{frame}

\begin{frame}
    test
\end{frame}
% \begin{frame}
%     - was passiert hier?
%     - gamma/hadron vergleich
%     - schauer bild mit den clustern
%     - importance für ML zeigen
% \end{frame}

\section{Number of islands}
\begin{frame}{Finding distinctive islands}
    \includegraphics[width=\linewidth]{images/islands.pdf}
    islands, toyshower on flash cam
\end{frame}
\begin{frame}
    - was ist tailcuts?
    - wo liegt der unterschied? -> besonders zeitkomponente
    - ist die schon richtig drin in cta?
    - cleaning level von tailcuts ableiten
    - vergleich auf schauerbildern
    - extremfälle: was bringt das?
    - Effekt auf ML
\end{frame}

\begin{frame}{Cleaning methods}
    \begin{columns}[T] % align columns
        \begin{column}{.48\textwidth}
            Tailcuts Cleaning
            \begin{enumerate}
                \item "two treshold procedure"
                \item pixels above t1 will be kept
                \item neighboring pixels above t2 will be kept
                \item "lonely" pixels wont survive
            \end{enumerate}
        \end{column}
        \begin{column}{.48\textwidth}
            FACT image cleaning
            \begin{enumerate}
                \item similar behaviour, but also uses information about the arrival times
                \item pixels with a very different arrival time than their neighbours get removed
                \item for now arrival times are integers in ctapipe -> limits search for optimal threshold
                \item removes "lonely" pixels multiple times
                \item one would assume less separated pixels
                \item intensity threshold should probably be a bit more loose than with tailcuts
        \end{enumerate}
        \end{column}
    \end{columns}
\end{frame}

\begin{frame}{timing information}
    bild mit intensity und peakpos
\end{frame}

\begin{frame}{comparison}
    links tailcut, rechts fact

    how does this affect separation and ergression models and hillas reconstruction?
\end{frame}


% vlt besser je config die ergebnisse?

\section{Machine learning impacts}

\begin{frame}
    \centering
    {\Huge \textbf{Machine learning impacts}}
\end{frame}

\begin{frame}{Setup and expectations}
    %\centering
    %{\textbf{Expectations}}
    \begin{itemize}
        \item A few \num{100000} diffuse gamma and proton MC events
        \item {Preprocessed with ctapipe, machine learning with
              aict-tools \\ {\footnotesize{\cite{aict}}}}
        \vspace{10pt}
        \item Tailcuts cleaning should perform pretty well with the chosen parameters
        \item Cleaning might affect separator performance
        \item Number of islands might contribute to separator performance
        \item Number of islands will probably not constribute to gamma energy regression
    \end{itemize}
\end{frame}

\begin{frame}{Gamma/Hadron Separation - AUC}
    \begin{columns}[T] % align columns
        \begin{column}{.48\textwidth}
            \textbf{Tailcuts cleaning:}
            \vspace{5pt}
            \begin{figure}
                \includegraphics[width=0.8\linewidth]{images/result_plots/tail1/sep_diff_1-crop.pdf}
            \end{figure}
        \end{column}
        \begin{column}{.48\textwidth}
            \textbf{FACT cleaning:}
            \vspace{5pt}
            \begin{figure}
                \includegraphics[width=0.8\linewidth]{images/result_plots/fact2/sep_diff_1-crop.pdf}
            \end{figure}
        \end{column}
    \end{columns}
\end{frame}

\begin{frame}{Gamma/Hadron Separation - Features}
    \begin{columns}[T] % align columns
        \begin{column}{.48\textwidth}
            \textbf{Tailcuts cleaning:}
            \vspace{5pt}
            \begin{figure}
                \includegraphics[width=\linewidth]{images/result_plots/tail1/sep_diff_4-crop2.png}
            \end{figure}
        \end{column}
        \begin{column}{.48\textwidth}
            \textbf{FACT cleaning:}
            \vspace{5pt}
            \begin{figure}
                \includegraphics[width=\linewidth]{images/result_plots/fact2/sep_diff_4-crop2.png}
            \end{figure}
        \end{column}
    \end{columns}
\end{frame}

\begin{frame}{Energy Regression}
    \begin{columns}[T] % align columns
        \begin{column}{.48\textwidth}
            \textbf{Tailcuts cleaning:}
            \vspace{5pt}
            \begin{figure}
                \includegraphics[width=0.8\linewidth]{images/result_plots/tail1/reg_diff_1-crop.pdf}
            \end{figure}
        \end{column}
        \begin{column}{.48\textwidth}
            \textbf{FACT cleaning:}
            \vspace{5pt}
            \begin{figure}
                \includegraphics[width=0.8\linewidth]{images/result_plots/fact2/reg_diff_1-crop.pdf}
            \end{figure}
        \end{column}
    \end{columns}
\end{frame}


\begin{frame}{Energy Regression - Features}
    \begin{columns}[T] % align columns
        \begin{column}{.48\textwidth}
            \textbf{Tailcuts cleaning:}
            \vspace{5pt}
            \begin{figure}
                \includegraphics[width=\linewidth]{images/result_plots/tail1/reg_diff_4-crop.pdf}
            \end{figure}
        \end{column}
        \begin{column}{.48\textwidth}
            \textbf{FACT cleaning:}
            \vspace{5pt}
            \begin{figure}
                \includegraphics[width=\linewidth]{images/result_plots/fact2/reg_diff_4-crop.pdf}
            \end{figure}
        \end{column}
    \end{columns}
\end{frame}


\end{document}
